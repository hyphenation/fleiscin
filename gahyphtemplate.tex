% `gahyph.tex': Hyphenation patterns for Irish
% Copyright (C) 2004 Kevin P. Scannell <scannell@slu.edu>
%
% This program is free software; you can redistribute it and/or
% modify it under the terms of the GNU General Public License
% as published by the Free Software Foundation; either version 2
% of the License, or (at your option) any later version.
% 
% This program is distributed in the hope that it will be useful,
% but WITHOUT ANY WARRANTY; without even the implied warranty of
% MERCHANTABILITY or FITNESS FOR A PARTICULAR PURPOSE.  See the
% GNU General Public License for more details.
% 
%%% ====================================================================
%%%  @TeX-hyphen-file{
%%%     author          = "Kevin P. Scannell",
%%%     version         = "1.0",
%%%     date            = "23 January 2004",
%%%     time            = "15:46:12 CST",
%%%     filename        = "gahyph.tex",
%%%     email           = "scannell@slu.edu",
%%%     codetable       = "ISO/ASCII",
%%%     keywords        = "TeX, hyphen, Irish, Gaeilge",
%%%     supported       = "yes",
%%%     abstract        = "Hyphenation patterns for Irish (Gaeilge)",
%%%     docstring       = "This file contains the hyphenation patterns
%%%                        for the Irish language",
%%%  }
%%% ====================================================================
%
%    See the web page  http://borel.slu.edu/fleiscin/index.html
%    for more information on how these patterns were generated.
%  
%    Please report incorrect hyphenations to the author at the 
%    email address above.
%%%%%%%%%%%%%%%%%%%%%%%%%%%%%%%%%%%%%%%%%%%%%%%%%%%%%%%%%%%%%%%%%%%%%%%%%%%%%
\message{Hyphenation patterns `gahyph.tex' Version 1.0 <2004/01/22>}
\catcode`^^c1=11 \lccode`^^c1=`^^e1 \uccode`^^c1=`^^c1    % \'A
\catcode`^^c9=11 \lccode`^^c9=`^^e9 \uccode`^^c9=`^^c9    % \'E
\catcode`^^cd=11 \lccode`^^cd=`^^ed \uccode`^^cd=`^^cd    % \'\I
\catcode`^^d3=11 \lccode`^^d3=`^^f3 \uccode`^^d3=`^^d3    % \'O
\catcode`^^da=11 \lccode`^^da=`^^fa \uccode`^^da=`^^da    % \'U
\catcode`^^e1=11 \lccode`^^e1=`^^e1 \uccode`^^e1=`^^c1    % \'a
\catcode`^^e9=11 \lccode`^^e9=`^^e9 \uccode`^^e9=`^^c9    % \'e
\catcode`^^ed=11 \lccode`^^ed=`^^ed \uccode`^^ed=`^^cd    % \'\i
\catcode`^^f3=11 \lccode`^^f3=`^^f3 \uccode`^^f3=`^^d3    % \'o
\catcode`^^fa=11 \lccode`^^fa=`^^fa \uccode`^^fa=`^^da    % \'u

\patterns{
}

% See the web page cited above for descriptions of these ambiguities.
%  2004-01-23:  24 of them
\hyphenation{
bhrachta�
mbrachta�
ch�int�
gc�int�
choirt�
gcoirt�
chreata�
gcreata�
dhoirte�
dhoirtear
dhoirt�
doirtear
ndoirte�
ndoirtear
ndoirt�
bhfuadar
fhuadar
ghorta�
ngorta�
l�amar
luadar
r�alta�
dtiom-�int�
thiom-�int�
}
