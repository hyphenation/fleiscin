% Hyphenation patterns for Irish
% Copyright (C) 2004 Kevin P. Scannell <scannell@slu.edu>
%
% This program is free software; you can redistribute it and/or
% modify it under the terms of the GNU General Public License
% as published by the Free Software Foundation; either version 2
% of the License, or (at your option) any later version.
% 
% This program is distributed in the hope that it will be useful,
% but WITHOUT ANY WARRANTY; without even the implied warranty of
% MERCHANTABILITY or FITNESS FOR A PARTICULAR PURPOSE.  See the
% GNU General Public License for more details.
% 
%%% ====================================================================
%%%  @TeX-hyphen-file{
%%%     author          = "Kevin P. Scannell",
%%%     version         = "1.0",
%%%     date            = "23 January 2004",
%%%     time            = "15:46:12 CST",
%%%     filename        = "gahyph.tex",
%%%     email           = "scannell@slu.edu",
%%%     codetable       = "ISO/ASCII",
%%%     keywords        = "TeX, hyphen, Irish, Gaeilge",
%%%     supported       = "yes",
%%%     abstract        = "Hyphenation patterns for Irish (Gaeilge)",
%%%     docstring       = "This file contains the hyphenation patterns
%%%                        for the Irish language",
%%%  }
%%% ====================================================================
%
%    I know of no written standards for hyphenation of Irish;
%    the patterns below were bootstrapped from an initial
%    database of hyphenated words extracted from actual
%    printed material.   This database was augmented through
%    the use of some "sed" scripts which added hyphens in obvious 
%    places: after standard prefixes, before morphological features
%    (verb endings, plural endings, etc.), and inside compound words.
%    I then performed a series of bootstrap steps using Patgen
%    and further refining with sed.
%  
%    The resulting hyphenation patterns are very much "etymological"
%    vs. "phonological".   As a consequence, they do not always
%    agree with hyphenations I've found in texts.  For instance:
%
%     These patterns             Corpus
%      "Ceilt-each"     vs.    "Ceil-teach"
%      "siosc-adh"      vs.    "sios-cadh" 
%      "craic-eann"     vs.    "crai-ceann"
%      "ceann-aithe"    vs.    "cean-naithe"
%      "tuairt-e�il"    vs.    "tuair-te�il"
%      "comh-alta"      vs.    "com-halta"
%
%    The last, of course, is an abomination of the worst kind.
%
%    Please report incorrect hyphenations to the author at the 
%    email address above.
%%%%%%%%%%%%%%%%%%%%%%%%%%%%%%%%%%%%%%%%%%%%%%%%%%%%%%%%%%%%%%%%%%%%%%%%%%%%%
\message{Hyphenation patterns `gahyph.tex' Version 1.0 <2004/01/22>}
\catcode`^^c1=11 \lccode`^^c1=`^^e1 \uccode`^^c1=`^^c1    % \'A
\catcode`^^c9=11 \lccode`^^c9=`^^e9 \uccode`^^c9=`^^c9    % \'E
\catcode`^^cd=11 \lccode`^^cd=`^^ed \uccode`^^cd=`^^cd    % \'\I
\catcode`^^d3=11 \lccode`^^d3=`^^f3 \uccode`^^d3=`^^d3    % \'O
\catcode`^^da=11 \lccode`^^da=`^^fa \uccode`^^da=`^^da    % \'U
\catcode`^^e1=11 \lccode`^^e1=`^^e1 \uccode`^^e1=`^^c1    % \'a
\catcode`^^e9=11 \lccode`^^e9=`^^e9 \uccode`^^e9=`^^c9    % \'e
\catcode`^^ed=11 \lccode`^^ed=`^^ed \uccode`^^ed=`^^cd    % \'\i
\catcode`^^f3=11 \lccode`^^f3=`^^f3 \uccode`^^f3=`^^d3    % \'o
\catcode`^^fa=11 \lccode`^^fa=`^^fa \uccode`^^fa=`^^da    % \'u

\patterns{
}
